\documentclass[11pt]{article}
\usepackage[utf8]{inputenc}
\usepackage[T1]{fontenc}
\usepackage{hyperref}
\usepackage{palatino}
\usepackage{paralist}
\usepackage{verbatim}

\usepackage[a4paper,top=2cm,bottom=2cm,left=2.5cm,right=2.5cm]{geometry}

\makeatletter
\def\verbatim@font{\ttfamily\small}
\makeatother

\usepackage{minted}
\newminted{clojure}{fontsize=\footnotesize,frame=lines,linenos}


\title{Solving the TTC 2013 Flowgraphs Case with FunnyQT}
\author{Tassilo Horn\\
  \href{mailto:horn@uni-koblenz.de}{horn@uni-koblenz.de}\\
  Institute for Software Technology\\
  University Koblenz-Landau}

\clubpenalty = 10000
\widowpenalty = 10000
\displaywidowpenalty = 10000

%% Reduce the space between image and captions
\setlength\abovecaptionskip{0.1cm}
\setlength\belowcaptionskip{0cm}


\begin{document}

\maketitle

\begin{abstract}
  This paper describes the FunnyQT solution to the TTC 2013 Flowgraphs
  Transformation Case.

  FunnyQT is a model querying and model transformation library for the
  functional Lisp-dialect Clojure.  It supports the modeling frameworks JGraLab
  and EMF natively, and it is designed to be extensible towards supporting
  other frameworks as well.

  FunnyQT provides a rich and efficient querying API, a polymorphic function
  API, a model manipulation API, and on top of those, there are several
  sub-APIs for implementing several kinds of transformations such as ATL-like
  model transformations or programmed graph transformations.

  For solving this case, the model transformation API and the polymorphic
  function API have been used for the JaMoPP to structure graph transformation.
  The control and data flow analysis is performed algorithmically using
  FunnyQT's plain querying and model manipulation APIs.  The final task
  requesting a simple DSL for validating the result models has been tackled by
  combining FunnyQT's querying API and Clojure metaprogramming.
\end{abstract}

\section{Introduction}
\label{sec:introduction}

\emph{FunnyQT} is a new model querying and transformation approach.  Instead of
inventing yet another language with its own concrete syntax and semantics, it
is implemented as an API for the functional, JVM-based Lisp-dialect
Clojure\footnote{\url{http://clojure.org/}}.  It's JVM-basing provides
wrapper-free access to all existing Clojure and Java libraries, and to other
tools in the rich Java ecosystem such as profilers.

FunnyQT natively supports the de-facto standard modeling framework EMF
\cite{Steinberg2008EEM} and the TGraph modeling framework
JGraLab\footnote{\url{http://jgralab.uni-koblenz.de}}, and it is designed to be
extensible towards other frameworks as well.

FunnyQT's API is split up in several sub-APIs.  On the lowest level there is a
core API for any supported modeling framework providing functions for loading
and storing models, accessing, creating, and deleting model elements, and
accessing and setting attribute values.  These core APIs mainly provide a
concise and expressive interface to the native Java APIs of the frameworks.  On
top of that, there's a generic API providing the subset of core functionality
that is common to both supported frameworks such as navigation via role names,
access to and manipulation of element properties, or functionality concerned
with typing imposed by metamodels.  Furthermore, there is a generic quering API
providing important querying concepts such as quantified expressions, regular
path expressions, or pattern matching.

Based on those querying and model manipulation APIs, there are several sub-APIs
for implementing different kinds of transformations.  For example, there is a
model transformation API similar to ATL \cite{ATL05} or ETL
\cite{booklet:epsilon}, or there is an in-place transformation API for writing
programmed graph transformations similar to GrGen.NET
\cite{manual:GrGenManual}.

Especially the pattern matching API and the transformation APIs make use of
Clojure's Lisp-inherited metaprogramming facilities
\cite{Graham1993OnLisp,Hoyte08LoL} in that they provide macros creating
internal DSLs \cite{book:Fowler2010DSL} providing concise, boilerplate-free
syntaxes to users.  Patterns and transformations written in these internal DSLs
get transformed to usual Clojure code using the FunnyQT querying and model
manipulation APIs by the Clojure compiler.

For solving the tasks of this transformation case, FunnyQT's model
transformation API and its polymorphic function API have been used for
converting the JaMoPP models into simpler structure graphs (Task 1,
Section~\ref{sec:task-1}).  Both the control flow analylis (Task 2,
Section~\ref{sec:task-2}) and the data flow analysis (Task 3,
Section~\ref{sec:task-3}) have been tackled algorithmically using FunnyQT's
plain querying and model manipulation APIs.  The task of validating the control
and data flow links using a simple DSL (Task 4, Section~\ref{sec:task-4}) has
been solved by using FunnyQT's querying API and Clojure metaprogramming.


\section{Task 1: JaMoPP to StructureGraph}
\label{sec:task-1}

According to the case description \cite{flowgraphcasedesc}, the goal of this
task is to transform a fine-granular Java syntax graph conforming to the JaMoPP
metamodel \cite{jamopp09} into a much simpler structure graph model that only
contains statements and expressions that are neither structured nor subdivided
any further.  However, the original Java code of these statements and
expressions should be reflected in the new elements' \verb|txt| attribute.
This model-to-text transformation is described in
Section~\ref{sec:jamopp-text}.  Thereafter, the model-to-model transformation
creating a structure graph from a JaMoPP model is described in
Section~\ref{sec:jamopp-struct-graph}.


\subsection{JaMoPP to Text}
\label{sec:jamopp-text}

This model-to-text transformation is implemented using FunnyQT's polymorphic
function API.  A polymorphic function is a function that is declared once, and
then arbitrary many implementations for concrete metamodel types can be added.
When a polymorphic function is called, the actual implementation is determined
similarly to the typical dispatch in object-oriented programming languages.  If
there's no implementation provided for the element's type or one of its
supertypes, an exception is thrown.

The function \verb|stmt2str| implements the model-to-text transformation
required for solving task 1.  It is declared as follows.

\begin{clojurecode}
(declare-polyfn stmt2str [elem])
\end{clojurecode}

\verb|declare-polyfn| declares a new polymorphic function.  Its name is
\verb|stmt2str|, and it receives exactly one parameter \verb|elem|.  It's task
is to create a string representation matching the concrete Java syntax for the
provided JaMoPP model element.

After the polymorphic function has been declared, implementations for concrete
metamodel types can be added using \verb|defpolyfn|.  For example, this is the
implementation for JaMoPP elements of type \verb|ClassMethod|:

\begin{clojurecode}
(defpolyfn stmt2str 'members.ClassMethod
  [method]
  (str (eget method :name) "()"))
\end{clojurecode}

I.e., to convert a JaMoPP method element to a string, the method's name is
concatenated with a pair of parentheses.

For structured elements, the implementations simply call the polymorphic
function for their children concatenating the results.  For example, this is
the polymorphic function for converting elements of type
\verb|AssignmentExpression| to a string.

\begin{clojurecode}
(defpolyfn stmt2str 'expressions.AssignmentExpression
  [ae]
  (str (stmt2str (eget ae :child)) " "
       (stmt2str (eget ae :assignmentOperator)) " "
       (stmt2str (eget ae :value))))
\end{clojurecode}

The \verb|child| of the assignment expression is some variable, the assignment
operator is one of \verb|=|, \verb|+=|, \verb|-=|, \verb|*=|, or \verb|/=|, and
\verb|value| is an arbitrary expression.  These three components are converted
to strings and then concatenated.

Basically, there could be one polymorphic function per JaMoPP type occuring in
one of the input models.  However, to save some lines of code, some
implementations are defined for some superclass and dispatch on concrete
subclasses themselves.  One of these implementations is that for operators
which will be invoked for the \verb|assignmentOperator| in the previous
listing:

\begin{clojurecode}
(defpolyfn stmt2str 'operators.Operator
  [op]
  (type-case op
    'operators.Multiplication "*"
    'operators.Subtraction    "-"
    'operators.Addition       "+"
    'operators.Division       "/"
    'operators.LessThan       "<"
    'operators.GreaterThan    ">"
    'operators.Assignment     "="
    'operators.MinusMinus     "--"
    'operators.PlusPlus       "++"
    'operators.AssignmentPlus "+="
    'operators.Equal          "=="))
\end{clojurecode}

\verb|type-case| gets some model element and several clauses.  Each clause
consists of a metamodel type and an expression.  The expression paired with the
first matching metamodel type is evaluated to form the result of the
\verb|type-case|.  If no clause matches, an exception is thrown.  As can be
seen, there are several Java operators missing (e.g., \verb|AssignmentMinus|
aka \verb|-=|), because they don't occur in the provided input models.

All in all, the polymorphic \verb|stmt2str| function consists of 22
implementations for various JaMoPP metamodel types accounting to a total of
about 110 lines of code.


\subsection{JaMoPP to Structure Graph}
\label{sec:jamopp-struct-graph}

The model-to-model transformation part of task 1 is implemented using FunnyQT's
ATL- or ETL-like model transformation API.  It realizes the JaMoPP to structure
graph transformation requested by task 1, and it also creates the vars and
params important for the data flow analysis as requested by task 3
(Section~\ref{sec:task-3}).

Before going into the details of the transformation, one helper function
\verb|used-vars| needs to be introduced.

\begin{clojurecode}
(defn used-vars [s]
  (reachables s [p-seq [p-* <>--]
                 [p-restr 'references.IdentifierReference]
                 :target]))
\end{clojurecode}

It takes a JaMoPP statement \verb|s| and returns the set of variables used
inside this statement.  \verb|reachables| gets an element and regular path
expression specifying the structure of paths that may be traversed.  Here, the
path expression specifies that from \verb|s| zero or many (\verb|p-*|)
containment references from container to child (\verb|<>--|) may be traversed.
The intermediate results are then filtered to elements of type
\verb|IdentifierReference|, and from those the \verb|target| reference is
traversed leading to a variable.

The actual transformation starts by defining its name and input and output
models.

\begin{clojurecode*}{firstnumber=5}
(deftransformation java2flowgraph [[in :emf] [out :emf]]
\end{clojurecode*}

There could be arbitrary many input and output models, and they could be of
different kinds, e.g., a transformation could receive a JGraLab TGraph and some
EMF model, and create an output EMF model.  Here, it gets only the JaMoPP EMF
input model which is bound the the variable \verb|in|, and one single structure
graph output model bound to \verb|out|.

In the body of such a transformation, arbitrary rules may be declared.  The
first rule is the \verb|method2method| rule shown in the next listing.  The
\verb|^:top| metadata preceeding the rule name specifies that the rule is a
top-level rule.  Such rules are applied to all matching elements by the
transformation itself, whereas non-top-level rules have to be called explicitly
from a top-level rule (directly or indirectly).

\begin{clojurecode*}{firstnumber=6}
  (^:top method2method [m]
      :from 'members.ClassMethod
      :when-let [mstmts (seq (eget m :statements))]
      :to [fgm 'flowgraph.Method, ex 'flowgraph.Exit]
      (eset! fgm :txt (stmt2str m))
      (eset! ex :txt "Exit")
      (eset! fgm :stmts (map stmt2item mstmts))
      (eset! fgm :exit ex)
      (eset! fgm :def (map param2param (eget m :parameters))))
\end{clojurecode*}

It receives a model element \verb|m| which must be of type \verb|ClassMethod|
in order for the rule to be applicable.  The \verb|:when-let| clause states
that this rule is only applicable to methods that contain
statements\footnote{\textsf{(seq coll)} is the canonical non-emptyness check in
  Clojure.}.  If it does, these statements are bound to \verb|mstmts|.  The
reason for this restriction is that the input JaMoPP models contain many
methods of classes in the Java standard library, but those don't have
statements because they were not parsed from source code.  The only method
parsed from code is the single test method contained in the provided Java
files.

The \verb|:to| clause declares the objects to be created.  Here, for a given
JaMoPP method, a corresponding flowgraph method and its exit element are
created.

The remainder of the rule is its body.  Here, the \verb|txt| attribute of the
new method and its exit are set, the former using the polymorphic
\verb|stmt2str| function discussed in Section~\ref{sec:jamopp-text}.

The method's \verb|stmts| reference is set by applying another rule,
\verb|stmt2item|, to the statements of the JaMoPP method.  The higher-order
function \verb|map| takes a function and a collection and applies the function
to every element of the collection returning a sequence of function results.

Similarly, the method's parameters are transformed by mapping them to the
\verb|param2param| rule for setting the method's \verb|def| reference.

When a rule is called and is applicable, the elements specified in the
\verb|:to| clause are created, the body is evaluated, a mapping from input
element to output elements is created, and a vector containing the newly
created elements in the order of their declaration in \verb|:to| is returned.
As a special case, if a rule creates only one target element, the mapping is
from input element to that output element.

When the rule is applied another time for the same input element, it simply
returns the result of the first application without creating anything new.

A special kind of rules are generalizing rules such as the one shown in the
next listing.

\begin{clojurecode*}{firstnumber=15}
  (stmt2item [stmt]
      :generalizes [local-var-stmt2simple-stmt condition2if block2block
                    return2return while-loop2loop break2break continue2continue
                    label2label stmt2simple-stmt])
\end{clojurecode*}

This is quite similar to mapping disjunction in QVT Operational Mappings.  When
this rule is called, the rules specified in the \verb|:generalizes| vector are
tried one after the other, and the first applicable one is applied and its
result is returned.  The rules are ordered from most specific to most lax, and
the last rule, \verb|stmt2simple-stmt| is a catch-all rule applicable to any
JaMoPP statement.

Since it is also the most complex rule, it is depicted in the next listing.

\begin{clojurecode*}{firstnumber=19}
  (stmt2simple-stmt [s]
      :from 'statements.Statement
      :to [fgss 'flowgraph.SimpleStmt]
      (eset! fgss :txt (stmt2str s))
      (doseq [aex  (reachables s [p-seq [p-* <>--]
                                  [p-restr 'expressions.AssignmentExpression]])]
        (eadd! fgss :def (var-creating-rule (the (used-vars (adj aex :child)))))
        (eaddall! fgss :use (map var-creating-rule (used-vars (adj aex :value)))))
      (doseq [umex (reachables s [p-seq [p-* <>--]
                                  [p-restr 'expressions.UnaryModificationExpression]])]
        (let [var (var-creating-rule (the (used-vars (adj umex :child))))]
          (eadd! fgss :def var)
          (eadd! fgss :use var))))
\end{clojurecode*}

It receives a JaMoPP statement \verb|s|, creates a new \verb|SimpleStmt|
\verb|fgss| in the target model, and sets its \verb|txt| attribute using the
polymorphic \verb|stmt2str| function.

The remainder of the rule deals with setting the new simple statement's
\verb|def| and \verb|use| references.  If the statement contains assignment
expressions such as \verb|a = b + c|, the flowgraph var corresponding to the
assigned variable of each such expression is added to the statement's
\verb|def| reference.  \verb|var-creating-rule| is another rule generalizing
the two rules transforming JaMoPP method parameters and local variables to
flowgraph \verb|Param| and \verb|Var| objects, respectively.  The \verb|def|
reference is set to the flowgraph vars corresponding to the variables used in
the \verb|value| expression of the assignment expression.  Because the regular
path expression in line 23 uses the zero-or-many iteration \verb|p-*|, it might
also be that \verb|s| doesn't contain an assignment statement but it is an
assignment statement itself.

Likewise, if the statement \verb|s| contains or is a
\verb|UnaryModificationExpression| such as \verb|i++|, the \verb|def| and
\verb|use| references are set to the flowgraph var element corresponding to the
JaMoPP variable being modified.

The complete \verb|java2flowgraph| model-to-model transformation consists of 15
rules with 89 lines of code in total.



\section{Task 2: Control Flow Analysis}
\label{sec:task-2}

The sole purpose of this task is to create \verb|cfNext| links between
\verb|FlowInstr| elements in the flowgraph model created by the model-to-model
transformation realizing task 1.  Every such flow instruction should be
connected to every other flow instruction that may be the next one in the
program's control flow.  This challenge has been tackled algorithmically using
FunnyQT's plain quering and model manipulation APIs.

The idea of the algorithm is quite easy.  It uses a sequence of statements as
intermediate representation to work on realizing a pre-order depth-first
iteration with look-ahead through the method's statements.  In the general
case, every flow instruction in that sequence has to be connected with the
immediately following flow instruction in the sequence.  For various kinds of
statements, special rules are needed.  For example, when encountering a block
in the sequence (which is no flow instruction), the the block is replaced with
its contents.

Since the next statement in the sequence might not be a flow instruction but
some structured statement like a block, an if-statement, or a loop, there's the
helper function \verb|cf-peek|.

\begin{clojurecode}
(defn cf-peek [el]
  (if (has-type? el 'flowgraph.FlowInstr)
    el
    (recur (first (econtents el)))))
\end{clojurecode}

It receives some element \verb|el|.  If that element is a flow instruction, it
is simply returned.  Else, the function calls itself recursively with the first
element of its contents, thus delivering the first flow instruction of a
possibly nested composite statement.  For example, when it is called for a
label containing a block whose first statement is a loop, it will return the
the expression which is the loop's condition.

The function synthesizing the control flow links is named \verb|cf-synth| using
the algorithm sketched above will be explained in the next listings.  It
receives the sequence of statements \verb|v|, the method's \verb|Exit| node
\verb|exit|, the current loop's \verb|loop-expr|, the statement following the
current loop \verb|loop-succ|, and a map \verb|label-succ-map| that assigns to
each label reachable in the current scope the statement following the labeled
statement.  The \verb|exit| parameter is used for handling return statements,
and the last three parameters are used for handling break and continue
statements.  Initially, the function is called with \verb|v| only containing
the method, and \verb|exit| bound to that method's exit.  All other parameters
are \verb|nil|.

\begin{clojurecode*}{firstnumber=5}
(defn cf-synth [v exit loop-expr loop-succ label-succ-map]
  (when (seq v)
    (let [[el & [n & _ :as tail]] v]
      (type-case el
\end{clojurecode*}

If the sequence \verb|v| is not empty, its first element is bound to \verb|el|,
and its rest is bound to \verb|tail|.  Furthermore, the first element of the
rest (i.e., the second element of the sequence) is bound to \verb|n|.  This
technique of binding components of collections by mimicing the collection's
structure is known as \emph{destructuring} in the Lisp-world.

After binding these elements, a \verb|type-case| dispatches on \verb|el|'s
metamodel type.  If the element is a method, a control flow link to that
method's first flow instruction is created, and the function recurses with the
method's statements.  \verb|recur| is a Clojure special form enabling a
tail-recursion that doesn't consume space on the call stack.  If \verb|recur|
doesn't occur in tail-position, the Clojure compiler will throw an exception.

\begin{clojurecode*}{firstnumber=9}
        'flowgraph.Method
                   (let [stmts (econtents el)]
                     (eadd! el :cfNext (cf-peek (first stmts)))
                     (recur stmts exit nil nil nil))
\end{clojurecode*}

If the current element is a simple statement and there's a next element in the
sequence, a control flow link is added to the first flow instruction of that
next statement.  Then, the function recurses with the tail of the sequence
keeping the remaining parameters as-is.

\begin{clojurecode*}{firstnumber=13}
        'flowgraph.SimpleStmt
                   (do (when n (eadd! el :cfNext (cf-peek n)))
                       (recur tail exit loop-expr loop-succ label-succ-map))
\end{clojurecode*}

If the current element is a block, the function simply recurses with that
block's statements prepended to the tail of the sequence keeping the remaining
parameters as-is.  That is, a block is replaced with its contents.

\begin{clojurecode*}{firstnumber=16}
        'flowgraph.Block
                   (recur (concat (econtents el) tail)
                          exit loop-expr loop-succ label-succ-map)
\end{clojurecode*}

If the current element is an expression of some if- or loop-statement and
there's a next statement in the sequence, a control flow link is added to the
first flow instruction of that next statement.  Again, the function recurses
with the tail of the sequence keeping the remaining parameters as-is.

\begin{clojurecode*}{firstnumber=19}
        'flowgraph.Expr
                   (do (when n (eadd! el :cfNext (cf-peek n)))
                       (recur tail exit loop-expr loop-succ label-succ-map))
\end{clojurecode*}

If the current element is a label, the function recurses with that label's
statement prepended to the tail of the sequence.  A mapping from this label to
the statement following it is added to the \verb|label-succ-map|.  This
statement's first flow instruction is where the control flow continues when
breaking to this label.

\begin{clojurecode*}{firstnumber=22}
        'flowgraph.Label
                   (recur (cons (eget el :stmt) tail) exit loop-expr loop-succ
                          (assoc label-succ-map el n))
\end{clojurecode*}

If the current element is a return statement, a control flow link is added to
the exit node of the method.  Thereafter, the function recurses again with
unchanged parameters.

\begin{clojurecode*}{firstnumber=25}
        'flowgraph.Return
                   (do (eadd! el :cfNext exit)
                       (recur tail exit loop-expr loop-succ label-succ-map))
\end{clojurecode*}

If the current element is a break statement, two cases have to be
destinguished.  If the break is labled, a control flow link is added to the
first flow instruction of the statement following the label.  The statement
following the label is looked up in the \verb|label-succ-map|.

If the break is not labled, a control flow link is added to the first flow
instruction in the statement following the surrounding loop which is bound to
\verb|loop-succ|.

In any case, the function recurses with the tail of the sequence keeping all
other parameters as-is.

\begin{clojurecode*}{firstnumber=28}
        'flowgraph.Break
                   (do (if-let [l (eget el :label)]
                         (eadd! el :cfNext (cf-peek (label-succ-map l)))
                         (eadd! el :cfNext (cf-peek loop-succ)))
                       (recur tail exit loop-expr loop-succ label-succ-map))
\end{clojurecode*}

If the current element is a continue statement, the same two cases have to be
distinguished.  In case of a labeled continue, a control flow link to the first
flow instruction of that label is added.  Because Java allows only to continue
to labeled loops, this flow instruction is the expression of that loop's
condition.

If the continue is not labeled, a control flow link to the nearest surrounding
loop's expression, \verb|loop-expr|, is added.

\begin{clojurecode*}{firstnumber=33}
        'flowgraph.Continue
                   (do (if-let [l (eget el :label)]
                         (eadd! el :cfNext (cf-peek l))
                         (eadd! el :cfNext loop-expr))
                       (recur tail exit loop-expr loop-succ label-succ-map))
\end{clojurecode*}

If the current element is a loop, this loop's condition is bound to
\verb|expr|, and its body is bound to \verb|body|.  The function recurses with
the expression, the body, and again the expression prepended to the tail of the
sequence.  The effect is that there will be a control flow link from the
expression to the first flow instruction in the body (the true-case of the loop
condition), control flow links from the final non-return/break/continue flow
instructions in the body to the expression (the repeation), and a control flow
link from the expression to the first flow instruction following the loop (the
false-case of the loop condition).  In the recursive call, the current loop
expression \verb|loop-expr| is bound to \verb|expr|, and the statement
following the loop \verb|loop-succ| is bound to \verb|n|, thus establishing a
new scope for contained and non-labeled break and continue statements.

\begin{clojurecode*}{firstnumber=38}
        'flowgraph.Loop
                   (let [[expr body] (econtents el)]
                     (recur (cons expr (cons body (cons expr tail)))
                            exit expr n label-succ-map))
\end{clojurecode*}

If the current element is an if-statement, its condition is bound to
\verb|expr|, its then-statement is bound to \verb|then|, and its else-statement
(which might be \verb|nil|) is bound to \verb|else|.

First the function recurses with a vector containing only the expression, the
then-statement, and the first flow instruction in the statement following the
if-statement.  The effect is that a control flow link will be added from the
expression to the first flow instruction in the then-statement, and the final
non-return/break/continue flow instructions in the then-statement will be
connected to the first flow instruction following the if-statement.  Note that
this call is done by calling the function itself rather than \verb|recur|,
because the call is not in tail-position.  As a result, deeply nested
if-statements could produce stack-overflows in theory.  However, the nesting
would need to exceed at least 3000 levels with standard JVM settings in order.

\begin{sloppypar}
  Depending on wether the if-statement has an else-part, the function then
  recurses either with the expression and else-statement prepended to the tail,
  or with only the expression prepended to the tail.  In the first case, the
  effect is that a control flow link will be created from the expression to the
  first flow instruction in the else-statement, and the final
  non-return/break/continue flow instructions in the else-statement will be
  connected to the first flow instruction following the if-statement.  In the
  other case, a control flow link will be created from the expression to the
  first flow instruction following the if-statement.
\end{sloppypar}

\begin{clojurecode*}{firstnumber=42}
        'flowgraph.If
                   (let [[expr then else] (econtents el)]
                     (cf-synth [expr then (cf-peek n)]
                               exit loop-expr loop-succ label-succ-map)
                     (if else
                       (recur (cons expr (cons else tail))
                              exit loop-expr loop-succ label-succ-map)
                       (recur (cons expr tail)
                              exit loop-expr loop-succ label-succ-map)))
\end{clojurecode*}

Finally, if the current element is the method's exit node, a sanity check is
performed.  At this point, the sequence \verb|v| should be exhausted, so there
must be no next statement \verb|n|.

\begin{clojurecode*}{firstnumber=51}
        'flowgraph.Exit (assert (nil? n))))))
\end{clojurecode*}

The entry point to the control flow transformation is the function
\verb|synthesize-cf-edges| shown in the next listing.  It receives a flowgraph
model and calls the \verb|cf-synth| function for any method contained in the
model providing the method's exit node via its \verb|exit| parameter.

\begin{clojurecode*}{firstnumber=52}
(defn synthesize-cf-edges [model]
  (doseq [m (eallobjects model 'flowgraph.Method)
          :let [exit (the (eallobjects model 'flowgraph.Exit))]]
    (cf-synth [m] exit nil nil nil)))
\end{clojurecode*}

The control flow transformation consists of 55 lines of code and has been
completely printed and discussed in this section.


\section{Task 3: Data Flow Analysis}
\label{sec:task-3}

The purpose of this task is to create \verb|dfNext| links between
\verb|FlowInst| elements where the target element is a control flow successor
of the source element, the target element uses (reads) a variable that was
defined (written) by the source element, and the variable hasn't been rewritten
in between.  This definition has been implemented exactly as stated here,
because although it's not the most efficient algorithm for the task, it is very
clear and concise.

The function \verb|find-nearest-definers| receives a flow instruction \verb|fi|
and a variable \verb|uv| used by it, and it returns a vector of the nearest
control flow predecessors that define that variable.

\begin{clojurecode}
(defn find-nearest-definers [fi uv]
  (loop [preds (mapcat #(adjs % :cfPrev) (if (coll? fi) fi [fi]))
         r []
         known #{}]
    (if (seq preds)
      (let [definers (filter #(member? uv (eget % :def)) preds)
            others   (remove #(member? uv (eget % :def)) preds)]
        (recur (remove #(member? % known) (mapcat #(adjs % :cfPrev) others))
               (into r definers)
               (into known preds)))
      r)))
\end{clojurecode}

In Clojure, \verb|loop| and \verb|recur| implement a local tail-recursion, that
is, inside a \verb|loop| a \verb|recur| form recurses not to the surrounding
function but to the surrounding \verb|loop|.  Initially, \verb|preds| is bound
to the immediate control flow predecessors of \verb|fi|, the result variable
\verb|r| is bound to the empty vector, and \verb|known| is bound to the empty
set.

If there are no predecessors, the result \verb|r| is returned (the else-branch
of the if).  If there are control flow predecessors, those are sorted into
\verb|definers|, i.e., flow instructions that define \verb|uv|.  The other
predecessors are sorted into \verb|others|, i.e., flow instructions that don't
write \verb|uv|.

Then it is recursed to the surrounding \verb|loop|.  \verb|preds| is rebound to
those control flow predecessors of \verb|others| that aren't already known in
order not to recurse infinitely in case of control flow cycles, the result
vector \verb|r| is rebound to the current \verb|r| value plus the new
\verb|definers|, and \verb|known| is rebound to the union of the current
\verb|known| value and the current \verb|preds|.

The function \verb|synthesize-df-edges| shown in the next listing is the entry
point to the data flow analysis receiving the flowgraph model.  It performs a
nested iteration\footnote{Clojure's \textsf{doseq} is similar to Java's
  enhanced for-each loop, but it also allows for nested iterations.} over all
flow instructions, the variables used by them, and the definers of these used
variables.  A data flow link is added from every nearest definer to the flow
instruction.

\begin{clojurecode*}{firstnumber=12}
(defn synthesize-df-edges [model]
  (doseq [fi (eallobjects model 'flowgraph.FlowInstr)
          used-var (eget fi :use)
          nearest-definer (find-nearest-definers fi used-var)]
    (eadd! nearest-definer :dfNext fi))
  (doseq [v (vec (eallobjects model 'Var))]
    (edelete! v)))
\end{clojurecode*}

Lastly, all \verb|Var| elements in the model are deleted.  This isn't requested
by the task, but because the unit tests of the transformation also create
visualizations of the result models, a smaller model size with much fewer links
results in a nicer layout.

Like with the control flow transformation, the data flow transformation
consisting of 18 lines of code has been completely printed and discussed.


\section{Task 4: Control and Data Flow Validation}
\label{sec:task-4}

The goal of task 4 is to enable offloading testing effort for the
transformations solving tasks 1 to 3 to programmers knowing Java but not the
transformation language used.  Therefore, some simple DSL should be provided
that lets them write down the expected control and data flow links.  Some tool
should be able to read that DSL and validate it against some result model.
This tool should print all missing \verb|cfNext|/\verb|dfNext| links, i.e.,
links defined in the specification but not occuring in the result model, and it
should print all false links, i.e., links occuring in the model but not in the
specification.

Before digging into the FunnyQT solution's implementation of this task, the
next listing shows the specification of the \verb|Test0.java.xmi| model.

\begin{clojurecode*}{linenos=false}
(make-test test-fg-transform-test0 "models/Test0.java.xmi"
           #{["testMethod()"   "int a = 1;"]     ;; expected cfNext links
             ["int a = 1;"     "int b = 2;"]
             ["int b = 2;"     "int c = a + b;"]
             ["int c = a + b;" "a = c;"]
             ["a = c;"         "b = a;"]
             ["b = a;"         "c = a / b;"]
             ["c = a / b;"     "b = a - b;"]
             ["b = a - b;"     "return b * c;"]
             ["return b * c;"  "Exit"]}
           #{["int a = 1;"     "int c = a + b;"] ;; expected dfNext links
             ["int b = 2;"     "int c = a + b;"]
             ["int c = a + b;" "a = c;"]
             ["a = c;"         "b = a;"]
             ["a = c;"         "c = a / b;"]
             ["a = c;"         "b = a - b;"]
             ["b = a;"         "c = a / b;"]
             ["b = a;"         "b = a - b;"]
             ["c = a / b;"     "return b * c;"]
             ["b = a - b;"     "return b * c;"]})
\end{clojurecode*}

\verb|make-test| is the entry point to the validation DSL.  It creates a new
test case with the given name (e.g., \verb|test-fg-transform-test0|) which
tests some concrete model whose XMI file is provided.  Then, a set of expected
\verb|cfNext| and expected \verb|dfNext| links follow.  Every link is
represented as a vector containing the expected source and target flow
instructions, every flow instruction being represented as string of its
concrete Java syntax.  These strings identify the flow instructions
unambiguously, because in the test cases, no statement occurs more than once.

Alternatively, instead of providing the links as sets of tuples, only the
expected number of \verb|cfNext|/\verb|dfNext| links may be provided.  For the
larger models, writing down the expected control and data flow links is simply
not feasible.

\begin{clojurecode*}{linenos=false}
(make-test test-fg-transform-test9 "models/Test9.java.xmi" 14452 27202)
\end{clojurecode*}

Of course, here the validation might succeed although the model is wrong
anyway, however such a slight validation is still much better than no
validation at all.

\verb|make-test| is a \emph{macro}.  A macro is a function that will be called
by the Clojure compiler at compile-time.  It receives the unevaluated arguments
given to it, that is, its parameters are bound to code.  Clojure, like all
Lisps, is \emph{homoiconic}, meaning that code is represented using usual
Clojure data structures, e.g., lists, vectors, symbols, literals, etc.  Thus,
the macro is able to transform the code provided to it using standard Clojure
functions to some new bunch of code that takes its place.

Without going into details, simple macros usually use a syntax-quoted form as
the template for the code to generate, and inside such a template the
parameters of the macro can be inserted by unquoting them with a tilde.
Furthemore, new variables can be introduced by suffixing them with a hash.
Those variables get an automatically generated name that is guaranteed to be
unique and thus cannot clash with symbols contained in the code provided as
macro arguments.

\begin{clojurecode}
(defmacro make-test [n file expected-cfs expected-dfs]
  `(deftest ~n
     (let [fg-trg# (run-transformations ~file)
           exp-cfs# ~expected-cfs
           exp-dfs# ~expected-dfs
           cfs# (set (map (fn [[s# t#]] [(eget s# :txt) (eget t# :txt)])
                          (ecrosspairs fg-trg# :cfPrev :cfNext)))
           dfs# (set (map (fn [[s# t#]] [(eget s# :txt) (eget t# :txt)])
                          (ecrosspairs fg-trg# nil :dfNext)))]
       (cond
        (set? exp-cfs#) (let [cf-d1# (clojure.set/difference exp-cfs# cfs#)
                              cf-d2# (clojure.set/difference cfs# exp-cfs#)]
                          (is (empty? cf-d1#) "Missing cf-edges")
                          (is (empty? cf-d2#) "Too many cf-edges"))
        (number? exp-cfs#) (do
                             (println "Only checking number of cfNext links.")
                             (is (= exp-cfs# (count cfs#))))
        :else (println "No expected cfNext links given."))
       (cond
        (set? exp-dfs#) (let [df-d1# (clojure.set/difference exp-dfs# dfs#)
                              df-d2# (clojure.set/difference dfs# exp-dfs#)]
                          (is (empty? df-d1#) "Missing df-edges")
                          (is (empty? df-d2#) "Too many df-edges"))
        (number? exp-dfs#) (do
                             (println "Only checking number of dfNext links.")
                             (is (= exp-dfs# (count dfs#))))
        :else (println "No expected dfNext links given.")))))
\end{clojurecode}

In line 2, \verb|make-test| generates a new unit test with the name given as
parameter \verb|n|.  In line 3, the transformations are run
transformation\footnote{\textsf{run-transformations} is a function that takes a
  model file and applies the 3 transformations one after the other returning
  the final program dependence graph.  It also times the execution, saves the
  result model, and generates visualizations for smaller models.} and the
result model is bound to a variable.  Likewise, the expected links are bound to
variables.  Lines 6 and 8 bind the links actually contained in the model to
variables.  The FunnyQT function \verb|ecrosspairs| returns the sequence of
crosslinks restricted to those matching given source and target reference
names.  Each link is represented as a vector of source and target element.
Using \verb|map|, these links are converted to the form of the links in the
sets of expected links, i.e., every link is represented by a vector of the
source and target element's \verb|txt| attribute.

In lines 11 to 18 and lines 20 to 27, the differences between expected and
actual links and vice versa are calculated.  Both should be empty, else there
are either missing or too many links.  \verb|is| is similar to JUnit's
\verb|Assert.assertTrue()| method.  In case the expected links were provided as
a number, only actual number of links is validated against it.


\section{Running the Transformation on SHARE}
\label{sec:run-transformation}

The FunnyQT solution to this case (and the other cases) are installed on the
SHARE image \verb|Ubuntu12LTS_TTC13::FunnyQT.vdi|.  Running the solution is
simple.

\begin{compactenum}
\item Open a terminal.
\item Change to the Petri-Nets to Statecharts project:

  \verb|$ cd ~/Desktop/FunnyQT_Solutions/ttc-2013-flowgraphs/|
\item Run the test cases:

  \verb|$ lein test|
\end{compactenum}

This will run the complete transformation (JaMoPP to structure graph with vars,
control flow analysis, data flow analysis, validation) on all provided test
JaMoPP models and print the execution times.  The result models and
visualizations of the main test cases' results are saved to the \verb|results|
directory.

\section{Evaluation}
\label{sec:evaluation}



\bibliographystyle{alpha}
\bibliography{ttc13-funnyqt-flowgraph}


\end{document}




%%% Local Variables:
%%% mode: latex
%%% TeX-engine: pdflatex-shell-escape
%%% TeX-master: t
%%% End:
